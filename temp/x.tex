\documentclass[]{article}
\usepackage{lmodern}
\usepackage{amssymb,amsmath}
\usepackage{ifxetex,ifluatex}
\usepackage{fixltx2e} % provides \textsubscript
\ifnum 0\ifxetex 1\fi\ifluatex 1\fi=0 % if pdftex
  \usepackage[T1]{fontenc}
  \usepackage[utf8]{inputenc}
\else % if luatex or xelatex
  \ifxetex
    \usepackage{mathspec}
  \else
    \usepackage{fontspec}
  \fi
  \defaultfontfeatures{Ligatures=TeX,Scale=MatchLowercase}
\fi
% use upquote if available, for straight quotes in verbatim environments
\IfFileExists{upquote.sty}{\usepackage{upquote}}{}
% use microtype if available
\IfFileExists{microtype.sty}{%
\usepackage{microtype}
\UseMicrotypeSet[protrusion]{basicmath} % disable protrusion for tt fonts
}{}
\usepackage[margin=1in]{geometry}
\usepackage{hyperref}
\hypersetup{unicode=true,
            pdftitle={Diffusion Curve Generator Output},
            pdfborder={0 0 0},
            breaklinks=true}
\urlstyle{same}  % don't use monospace font for urls
\usepackage{graphicx,grffile}
\makeatletter
\def\maxwidth{\ifdim\Gin@nat@width>\linewidth\linewidth\else\Gin@nat@width\fi}
\def\maxheight{\ifdim\Gin@nat@height>\textheight\textheight\else\Gin@nat@height\fi}
\makeatother
% Scale images if necessary, so that they will not overflow the page
% margins by default, and it is still possible to overwrite the defaults
% using explicit options in \includegraphics[width, height, ...]{}
\setkeys{Gin}{width=\maxwidth,height=\maxheight,keepaspectratio}
\IfFileExists{parskip.sty}{%
\usepackage{parskip}
}{% else
\setlength{\parindent}{0pt}
\setlength{\parskip}{6pt plus 2pt minus 1pt}
}
\setlength{\emergencystretch}{3em}  % prevent overfull lines
\providecommand{\tightlist}{%
  \setlength{\itemsep}{0pt}\setlength{\parskip}{0pt}}
\setcounter{secnumdepth}{0}
% Redefines (sub)paragraphs to behave more like sections
\ifx\paragraph\undefined\else
\let\oldparagraph\paragraph
\renewcommand{\paragraph}[1]{\oldparagraph{#1}\mbox{}}
\fi
\ifx\subparagraph\undefined\else
\let\oldsubparagraph\subparagraph
\renewcommand{\subparagraph}[1]{\oldsubparagraph{#1}\mbox{}}
\fi

%%% Use protect on footnotes to avoid problems with footnotes in titles
\let\rmarkdownfootnote\footnote%
\def\footnote{\protect\rmarkdownfootnote}

%%% Change title format to be more compact
\usepackage{titling}

% Create subtitle command for use in maketitle
\newcommand{\subtitle}[1]{
  \posttitle{
    \begin{center}\large#1\end{center}
    }
}

\setlength{\droptitle}{-2em}

  \title{Diffusion Curve Generator Output}
    \pretitle{\vspace{\droptitle}\centering\huge}
  \posttitle{\par}
    \author{}
    \preauthor{}\postauthor{}
      \predate{\centering\large\emph}
  \postdate{\par}
    \date{09 January, 2019}


\begin{document}
\maketitle

\section{1. Introduction}\label{introduction}

These results are produced by diffusion curve generator hosted at
{[}insert URL{]} and developed by The University of Sheffield. The
method underpinning the calculations is based on a paper by Sabine Grimm
(see citation information at the end of this report)

The method takes probability distributions for three diffusion
parameters to generate curves based on the Bass Model of Product
Diffusion. The parameters, \(m\), \(N1\) and \(t'\), represent the
maximum number of adoptions attained, the number of adoptions in the
first year and the time at which the rate of diffusion reduces,
respectively.

\pagebreak

\section{2. Elicitation Input}\label{elicitation-input}

Input distributions

\begin{verbatim}
##      Expert                        M                   N1               t'
## 1: Expert A  Triangle(54.2, 10, 150)  Triangle(2.3, 0, 5) Normal(5.1, 1.5)
## 2: Expert B Triangle(158.8, 30, 230) Triangle(5.7, 2, 15) Normal(9.9, 1.5)
## 3: Expert C Triangle(204.4, 30, 410) Triangle(7.1, 2, 10) Normal(3.5, 1.1)
\end{verbatim}

Distributions of the input parameters (300 samples). The parameters were
sampled from mixture distributions.

\includegraphics{E:/Source/DCGenApp/temp/x_files/figure-latex/unnamed-chunk-2-1.pdf}

Summary of sample input parameters

\begin{verbatim}
##        M                N1                t         
##  Min.   : 14.54   Min.   : 0.5253   Min.   : 2.195  
##  1st Qu.: 76.83   1st Qu.: 3.0567   1st Qu.: 4.101  
##  Median :129.28   Median : 4.6671   Median : 5.350  
##  Mean   :145.07   Mean   : 5.2531   Mean   : 6.227  
##  3rd Qu.:196.47   3rd Qu.: 7.1450   3rd Qu.: 8.736  
##  Max.   :397.43   Max.   :14.1997   Max.   :12.133
\end{verbatim}

\pagebreak

\section{3. Parameter fitting}\label{parameter-fitting}

Distributions of the fitted parameters (300 samples).
\includegraphics{E:/Source/DCGenApp/temp/x_files/figure-latex/unnamed-chunk-4-1.pdf}

Summary of sample input parameters

\begin{verbatim}
##        M                p                   q         
##  Min.   : 14.54   Min.   :0.0001348   Min.   :0.6644  
##  1st Qu.: 76.83   1st Qu.:0.0014658   1st Qu.:0.9133  
##  Median :129.28   Median :0.0066614   Median :0.9904  
##  Mean   :145.07   Mean   :0.0091952   Mean   :0.9665  
##  3rd Qu.:196.47   3rd Qu.:0.0115413   3rd Qu.:1.0474  
##  Max.   :397.43   Max.   :0.0684298   Max.   :1.1403
\end{verbatim}

\pagebreak

\section{4. Generated diffusion
curves}\label{generated-diffusion-curves}

Generated diffusion curves. Statistics of mean and 70\% quantiles
showed.

\begin{itemize}
\tightlist
\item
  \(N(t)\) Number of cumulated adoptions at \(t\)
\item
  \(dN(t)\) New adoptions at \(t\)
\end{itemize}

\includegraphics{E:/Source/DCGenApp/temp/x_files/figure-latex/unnamed-chunk-6-1.pdf}

Mean curves

\begin{verbatim}
##  Time      N    dN
##     0   0.00  1.11
##     2   6.81  7.33
##     4  36.10 21.93
##     6  83.04 21.46
##     8 115.83 11.62
##    10 132.61  5.81
##    12 140.69  2.54
##    14 143.81  0.83
##    16 144.75  0.22
##    18 144.99  0.06
##    20 145.05  0.01
\end{verbatim}

\subsection{Reference}\label{reference}

Grimm SE, Stevens JW, Dixon S. Estimating Future Health Technology
Diffusion Using Expert Beliefs Calibrated to an Established Diffusion
Model. Value Health. 2018 Aug;21(8):944-950.


\end{document}
